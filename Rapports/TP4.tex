\documentclass[12pt,a4paper]{article}
\usepackage[T1]{fontenc}
\usepackage[utf8]{inputenc}
\usepackage{lmodern}
\usepackage[frenchb]{babel}
\usepackage[left=1.5cm,right=1.5cm,top=2cm,bottom=2cm]{geometry}
\usepackage{minted}
\usepackage[usenames,dvipsnames]{xcolor}
\usepackage{fancyhdr}
\setminted{breaklines, frame=single}
\author{Hamza ABBAD\\ Master Systèmes Informatiques Intelligents}
\title{\textbf{Rapport de TP ASGBD} \\ TP $n^{\circ}$4}
\pagestyle{fancyplain}
\lhead{Hamza ABBAD}
\chead{201200006342}
\rhead{SII M1 - Groupe 2}
\begin{document}
\maketitle
\begin{enumerate}
    \item Un code PL/SQL permettant d'afficher le numéro de chaque chambre avec son
    service et le nombre de lits occupés et libres :
    \inputminted[firstline=4, lastline=18]{SQL}{../Scripts/TP4.sql}
    Pour augmenter les salaires des infirmiers on doit d'abord désactiver la contrainte de vérification :
    \inputminted[firstline=21, lastline=21]{SQL}{../Scripts/TP4.sql}
    Ensuite, il faut définir la procédure :
    \inputminted[firstline=22, lastline=29]{SQL}{../Scripts/TP4.sql}
    Enfin, l'exécuter :
    \inputminted[firstline=30, lastline=30]{SQL}{../Scripts/TP4.sql}
    Pour vérifier les changements faites, on peut récupérer tout les salaires :
    \inputminted[firstline=31, lastline=31]{SQL}{../Scripts/TP4.sql}
\end{enumerate}
\end{document}
