\documentclass[12pt,a4paper]{article}
\usepackage[T1]{fontenc}
\usepackage[utf8]{inputenc}
\usepackage{lmodern}
\usepackage[frenchb]{babel}
\usepackage[left=1.5cm,right=1.5cm,top=2cm,bottom=2cm]{geometry}
\usepackage{framed}
\usepackage{minted}
\usepackage[usenames,dvipsnames]{xcolor}
\usepackage{fancyhdr}
\definecolor{shadecolor}{rgb}{0.97,0.97,0.97}
\setminted{breaklines, frame=single}
\author{Hamza ABBAD\\ Master Systèmes Informatiques Intelligents}
\title{\textbf{Rapport de TP ASGBD} \\ TP $n^{\circ}$4}
\pagestyle{fancyplain}
\lhead{Hamza ABBAD}
\chead{201200006342}
\rhead{SII M1 - Groupe 2}
\begin{document}
\maketitle
\textit{Toutes les requêtes dans ce TP sont exécutées en utilisant l'utilisateur qui a créé les tables dans le TP 1.}
\begin{enumerate}
    \item Un code PL/SQL permettant d'afficher le numéro de chaque chambre avec son
    service et le nombre de lits occupés et libres :
    \begin{snugshade}
        \inputminted[firstline=4, lastline=18]{SQL}{../Scripts/TP4.sql}
    \end{snugshade}
    Pour augmenter les salaires des infirmiers on doit d'abord désactiver la contrainte de vérification :
    \begin{snugshade}
        \inputminted[firstline=21, lastline=21]{SQL}{../Scripts/TP4.sql}
    \end{snugshade}
    Ensuite, il faut définir la procédure :\\
    \begin{snugshade}
        \inputminted[firstline=22, lastline=29]{SQL}{../Scripts/TP4.sql}
    \end{snugshade}
    Enfin, l'exécuter :
    \begin{snugshade}
        \inputminted[firstline=30, lastline=30]{SQL}{../Scripts/TP4.sql}
    \end{snugshade}
    Pour vérifier les changements faites, on peut récupérer tout les salaires :
    \begin{snugshade}
        \inputminted[firstline=31, lastline=31]{SQL}{../Scripts/TP4.sql}
    \end{snugshade}
    \item La procédure de vérification du salaire :
    \begin{snugshade}
        \inputminted[firstline=34, lastline=40]{SQL}{../Scripts/TP4.sql}
    \end{snugshade}
    Tester la procédure pour tout les infirmiers :
    \begin{snugshade}
        \inputminted[firstline=42, lastline=50]{SQL}{../Scripts/TP4.sql}
    \end{snugshade}
    Résultats :
    \begin{snugshade}
        \begin{verbatim}
HADJ Zouhir 18841.17 Vérification  positive
OUSSEDIK Hakim 17670.72 Vérification  positive
ABAD Abdelhamid 22470.32 Vérification  positive
ABAYAHIA Abdelkader 23611.88 Vérification  positive
ABBOU Mohamed 20373.68 Vérification  positive
ABDELOUAHAB OUAHIBA 21979.88 Vérification  positive
ABDEMEZIANE Madjid 18848.67 Vérification  positive
ACHAIBOU Rachid 26481.32 Vérification  positive
AGGOUN Khadidja 21372.58 Vérification  positive
AISSAT Salima 23581.26 Vérification  positive
BABACI Mourad 17678.22 Vérification  positive
BADI Hatem 31152.98 Vérification  négative
BAKIR ADEL 18944.42 Vérification  positive
BALI Malika 22476.32 Vérification  positive
BELABES Abdelkader 23620.88 Vérification  positive
BELHAMIDI Mustapha 20251.81 Vérification  positive
BELKACEMI Hocine 20322.68 Vérification  positive
BELKOUT Tayeb 21982.88 Vérification  positive
FERAOUN Houria 32599.71 Vérification  négative
CHAKER Nadia 26527.82 Vérification  positive
IGOUDJIL Redouane 21337.38 Vérification  positive
GHEZALI Lakhdar 22146.06 Vérification  positive
KOULA Brahim 23581.26 Vérification  positive
BELAID Layachi 20716.18 Vérification  positive
CHALABI Mourad 19494.82 Vérification  positive
SAIDOUNI Wafa 21227.81 Vérification  positive
Yalaoui Lamia 35255.71 Vérification  négative
AYATA Samia 19810.06 Vérification  positive

PL/SQL procedure successfully completed.
        \end{verbatim}
    \end{snugshade}
    \item Une fonction qui retourne pour chaque spécialité le nombre de medecins affectés :
    \begin{snugshade}
        \inputminted[firstline=53, lastline=59]{SQL}{../Scripts/TP4.sql}
    \end{snugshade}
    Exécuter la fonction pour chaque spécialitée pour afficher le nombre de medecins affectés à chaque spécialitée :
    \begin{snugshade}
        \inputminted[firstline=61, lastline=68]{SQL}{../Scripts/TP4.sql}
    \end{snugshade}
    Résultats :
    \begin{snugshade}
        \begin{verbatim}
Orthopédiste 5
Traumatologue 5
Pneumologue 5
Cardiologue 9
Radiologue 4
Anesthésiste 5

PL/SQL procedure successfully completed.
        \end{verbatim}
    \end{snugshade}
    \item Une procédure permettant d'ajouter un nouveau infirmier dans la base de données en faisant les vérifications nécessaires :
    \begin{snugshade}
        \inputminted[firstline=71, lastline=99]{SQL}{../Scripts/TP4.sql}
    \end{snugshade}
    Pour ajouter un infirmier il suffit d'exécuter cette procédure comme suit :
    \begin{snugshade}
        \inputminted[firstline=100, lastline=100]{SQL}{../Scripts/TP4.sql}
    \end{snugshade}
    S'il n'y a aucune erreur, les informations de l'infirmier seront automatiquement insérées dans les tables \texttt{EMPLOYE} et
    \texttt{INFIRMIER}, sinon, une erreur correspondant à la valeur erronée sera affichée.\\
    Quelques testes :
    \begin{snugshade}
        \begin{verbatim}
SQL> SELECT COUNT(*) FROM INFIRMIER;

  COUNT(*)
----------
        28

SQL> SELECT COUNT(*) FROM EMPLOYE;

  COUNT(*)
----------
        61

SQL> EXECUTE Ajouter_infirmier(325, 'ABBAD', 'Hamza', 'Cite 8 Mai 45',
'0551788241', 'CAR', 'JOUR', 25000);

PL/SQL procedure successfully completed.

SQL> SELECT COUNT(*) FROM INFIRMIER;

  COUNT(*)
----------
        29

SQL> SELECT COUNT(*) FROM EMPLOYE;

  COUNT(*)
----------
        62

SQL> SELECT * FROM EMPLOYE WHERE NUM_EMP = 325;

   NUM_EMP NOM_EMP
---------- --------------------------------------------------
PRENOM_EMP
--------------------------------------------------
ADRESSE_EMP
--------------------------------------------------------------------------------
TEL_EMP
----------
       325 ABBAD
Hamza
Cite 8 Mai 45
0551788241


SQL> SELECT * FROM INFIRMIER WHERE NUM_INF = 325;

   NUM_INF COD ROTA    SALAIRE
---------- --- ---- ----------
       325 CAR JOUR      25000

SQL> EXECUTE Ajouter_infirmier(15, 'XYZ', 'ABC', 'GHJK',
'2531490041', 'REA', 'JOUR', 17500);
Erreur : Un employe du même numéro existe déja !

PL/SQL procedure successfully completed.

SQL> EXECUTE Ajouter_infirmier(250, 'XYZ', 'ABC', 'GHJK',
'2531490041', 'DBA', 'JOUR', 17500);
Erreur : Le service spécifié n'existe pas !

PL/SQL procedure successfully completed.

SQL> EXECUTE Ajouter_infirmier(250, 'XYZ', 'ABC', 'GHJK',
'2531490041', 'REA', 'DEMI', 17500);
Erreur : La rotation spécifiée n'existe pas !

PL/SQL procedure successfully completed.

SQL> EXECUTE Ajouter_infirmier(250, 'XYZ', 'ABC', 'GHJK',
'2531490041', 'REA', 'JOUR', 7500);
Erreur : Le salaire doit être entre 10000 et 30000 !

PL/SQL procedure successfully completed.
        \end{verbatim}
    \end{snugshade}
    \end{enumerate}
\end{document}
