\documentclass[12pt,a4paper]{article}
\usepackage[T1]{fontenc}
\usepackage[utf8]{inputenc}
\usepackage{lmodern}
\usepackage[frenchb]{babel}
\usepackage[left=1.5cm,right=1.5cm,top=2cm,bottom=2cm]{geometry}
\usepackage{listings}
\usepackage[usenames,dvipsnames]{xcolor}
\usepackage{fancyhdr}
\author{Hamza ABBAD\\ Master Systèmes Informatiques Intelligents}
\title{\textbf{Rapport de TP ASGBD} \\ TP $n^{\circ}$2}
\lstdefinestyle{OracleSQL}{language=SQL,breaklines=true, frame=single, keywordstyle=\bfseries\color{Blue},
keywordstyle=[2]\color{Blue},stringstyle=\color{OliveGreen},extendedchars=false,sensitive=false,
commentstyle=\slshape\color{Gray},basicstyle=\small\ttfamily,
morekeywords={TABLESPACE,AUTOEXTEND,TEMPORARY,PRIVILEGES,DATAFILE,IDENTIFIED,TEMPFILE,TO,ONLINE,USER,NVARCHAR2,REFERENCES,
MODIFY,RENAME,DISABLE}
}
\pagestyle{fancyplain}
\lhead{Hamza ABBAD}
\chead{201200006342}
\rhead{SII M1 - Groupe 2}
\begin{document}
\maketitle
\begin{enumerate}
	\item Création de l'utilisateur \texttt{AdminHopital} :
	\lstinputlisting[firstline=2, lastline=5, style=OracleSQL]{../Scripts/TP2.sql}
	\item Quand on essaye de se connecter en utilisant cet utilisateur on reçoit l'erreur suivante :
	\begin{verbatim}
	ERROR:
	ORA-01045: user ADMINHOPITAL lacks CREATE SESSION privilege; logon denied
	\end{verbatim}
	C'est à dire que l'utilisateur n'a pas le droit de se connecter.
	\item Pour donner le droit de création des sessions, on doit se connecter à l'aide du compte \texttt{SYSTEM}
	et exécuter la requête suivante :
	\lstinputlisting[firstline=8, lastline=9, style=OracleSQL]{../Scripts/TP2.sql}
	\item Donner les privilèges de la création des tables et des utilisateurs (en utilisant l'utilisateur \texttt{SYSTEM}) :
	\lstinputlisting[firstline=12, lastline=13, style=OracleSQL]{../Scripts/TP2.sql}
	\item Cette requête cause l'erreur suivante :
	\begin{verbatim}
	SELECT * FROM Abbad.PATIENT
                    *
	ERROR at line 1:
	ORA-00942: table or view does not exist
	\end{verbatim}
	Le SGBD n'arrive pas a récupérer la table car l'utilisateur courant n'a pas le droit de la consulter.
	\item Donner les droits de lecture à l'utilisateur \texttt{AdminHopital} pour la table \texttt{PATIENT} :
	\lstinputlisting[firstline=25, lastline=26, style=OracleSQL]{../Scripts/TP2.sql}
	Maintenat on peut exécuter la requête :
	\lstinputlisting[firstline=28, lastline=32, style=OracleSQL]{../Scripts/TP2.sql}
	\item Quand on essaye de changer l'adresse du patient en utilisant l'utilisateur \texttt{AdminHopital}
	en utilisant cette requête :
	\lstinputlisting[firstline=35, lastline=35, style=OracleSQL]{../Scripts/TP2.sql}
	L'erreur suivante se produit :
	\begin{verbatim}
	UPDATE Abbad.PATIENT SET ADR_PAT = '152, rue Hassiba Ben Bouali 2ème étage 
	-Hamma-Alger' WHERE NOM_PATIENT = 'HADJ' AND 		PRENOM_PATIENT = 'Haroun'
    	         *
	ERROR at line 1:
	ORA-01031: insufficient privileges
	\end{verbatim}
	Ce qui signifie que cet utilisateur n'a pas le droit de faire une telle modification.
	\item A partir de l'utilisateur \texttt{SYSTEM}, on peut donner les droits de mise à jour à cet utilisateur
	pour la table \texttt{PATIENT} :
	\lstinputlisting[firstline=44, lastline=45, style=OracleSQL, extendedchars=true]{../Scripts/TP2.sql}
	Ensuite on peut exécuter la requête précédente à partir de l'utilisateur \texttt{AdminHopital}.
	\lstinputlisting[firstline=47, lastline=48, style=OracleSQL]{../Scripts/TP2.sql}
	\item L'utilisateur \texttt{AdminHopital} ne peut pas créer l'index, à l'exécution de cette requête :
	\lstinputlisting[firstline=51, lastline=51, style=OracleSQL]{../Scripts/TP2.sql}
	Il obtient l'erreur suivante :
	\begin{verbatim}
	CREATE INDEX PatientMutuelle_IX ON Abbad.PATIENT(MUTUELLE) TABLESPACE AdminHopital
	                                         *
	ERROR at line 1:
	ORA-01031: insufficient privileges
	\end{verbatim}
	Ce qui signifie que l'utilisateur n'a pas le privilège nécessaire pour créer les indexes.
	\item Pour autoriser l'utilisateur \texttt{AdminHopital} à créer les indexes, on exécute cette requête
	à partir de l'utilisateur \texttt{SYSTEM} :
	\lstinputlisting[firstline=60, lastline=60, style=OracleSQL]{../Scripts/TP2.sql}
	Réessayer à partir de \texttt{AdminHopital} :
	\lstinputlisting[firstline=62, lastline=63, style=OracleSQL]{../Scripts/TP2.sql}
	\item Enlever les privilèges précédemment accordés :
	\lstinputlisting[firstline=66, lastline=70, style=OracleSQL]{../Scripts/TP2.sql}
	\item Après l'exécution des requêtes précédentes par le \texttt{SYSTEM}, l'utilisateur \texttt{AdminHopital}
	ne poura plus créer des utilisateurs, tableaux ou indexes, et il ne poura ni voir ni modifier le contenu
	d'un tableau s'il était connecté pendant l'exécution requêtes précédentes, sinon, il ne peut même pas se connecter.
	\item Création du profil \texttt{Admin\_Profil} :
	\lstinputlisting[firstline=73, lastline=86, style=OracleSQL]{../Scripts/TP2.sql}
	\item Affectation du profil :
	\lstinputlisting[firstline=89, lastline=89, style=OracleSQL]{../Scripts/TP2.sql}
\end{enumerate}
\end{document}