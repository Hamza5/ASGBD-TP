\documentclass[12pt,a4paper]{article}
\usepackage[T1]{fontenc}
\usepackage[utf8]{inputenc}
\usepackage{lmodern}
\usepackage[frenchb]{babel}
\usepackage[left=1.5cm,right=1.5cm,top=2cm,bottom=2cm]{geometry}
\usepackage{listings}
\usepackage[usenames,dvipsnames]{xcolor}
\usepackage{fancyhdr}
\author{Hamza ABBAD\\ Master Systèmes Informatiques Intelligents}
\title{\textbf{Rapport de TP ASGBD} \\ TP $n^{\circ}$1}
\lstdefinestyle{OracleSQL}{language=SQL,breaklines=true, frame=single, keywordstyle=\bfseries\color{Blue},
keywordstyle=[2]\color{Blue},stringstyle=\color{OliveGreen},extendedchars=false,sensitive=false,
commentstyle=\slshape\color{Gray},basicstyle=\small\ttfamily,
morekeywords={TABLESPACE,AUTOEXTEND,TEMPORARY,PRIVILEGES,DATAFILE,IDENTIFIED,TEMPFILE,TO,ONLINE,USER,NVARCHAR2,REFERENCES,
MODIFY,RENAME,DISABLE}
}
\pagestyle{fancyplain}
\lhead{Hamza ABBAD}
\chead{201200006342}
\rhead{SII M1 - Groupe 2}
\begin{document}
\maketitle
\section{Création des TableSpaces et des utilisateurs}
	\begin{enumerate}
		\item
		\begin{enumerate}
			\item Requête : \lstinputlisting[language=SQL, firstline=2, lastline=2, style=OracleSQL]{../Scripts/TP1.sql}
				Résultat : \texttt{Tablespace created.}
			\item Requête : \lstinputlisting[language=SQL, firstline=3, lastline=3, style=OracleSQL]{../Scripts/TP1.sql}
			Résultat : \texttt{Tablespace created.}
		\end{enumerate}
		\item Requête : \lstinputlisting[language=SQL, firstline=4, lastline=4, style=OracleSQL]{../Scripts/TP1.sql}
			Résultat : \texttt{User created.}
		\item Requête : \lstinputlisting[language=SQL, firstline=5, lastline=5, style=OracleSQL]{../Scripts/TP1.sql}
			Résultat : \texttt{Grant succeeded.}
	\end{enumerate}
\section{Langage de définition de données}
	\begin{enumerate}
		\setcounter{enumi}{3}
		\item Les clés étrangères :\\
			\renewcommand{\arraystretch}{1.2}
			\begin{tabular}{|*{4}{p{4cm}|}}
			\hline
			\multicolumn{2}{|c|}{Tableau source} & \multicolumn{2}{|c|}{Tableau destination} \\ \hline
			Dans le tableau & colonne & Vers le tableau & colonne \\ \hline
			SERVICE & DIRECTEUR & MEDECIN & NUM-MED \\ \hline
			CHAMBRE & CODE-SERVICE & SERVICE & CODE-SERVICE \\ \hline
			CHAMBRE & SURVEILLANT & INFIRMIER & NUM-INF \\ \hline
			MEDECIN & NUM-MED & EMPLOYE & NUM-EMP \\ \hline
			INFIRMIER & CODE-SERVICE & SERVICE & CODE-SERVICE \\ \hline
			INFIRMIER & NUM-INF & EMPLOYE & NUM-EMP \\ \hline
			HOSPITALISATION & NUM-PATIENT & PATIENT & NUM-PATIENT \\ \hline
			HOSPITALISATION & CODE-SERVICE & SERVICE & CODE-SERVICE \\ \hline
			HOSPITALISATION & CODE-SERVICE, NUM-CHAMBRE & CHAMBRE & CODE-SERVICE \\ \hline
			SOIGNE & NUM-PATIENT & PATIENT & NUM-PATIENT \\ \hline
			\end{tabular}
		\item Création des relations
			\begin{enumerate}
				\item Relation \texttt{SERVICE}
					\lstinputlisting[firstline=11, lastline=16, style=OracleSQL]{../Scripts/TP1.sql}
					Résultat : \texttt{Table created.}
				\item Relation \texttt{CHAMBRE}
					\lstinputlisting[firstline=18, lastline=23, style=OracleSQL]{../Scripts/TP1.sql}
					Résultat : \texttt{Table created.}
				\item Relation \texttt{EMPLOYE}
					\lstinputlisting[firstline=25, lastline=31, style=OracleSQL]{../Scripts/TP1.sql}
						Résultat : \texttt{Table created.}
				\item Relation \texttt{MEDECIN}
					\lstinputlisting[firstline=33, lastline=36, style=OracleSQL]{../Scripts/TP1.sql}
						Résultat : \texttt{Table created.}
				\item Relation \texttt{INFIRMIER}
					\lstinputlisting[firstline=38, lastline=43, style=OracleSQL]{../Scripts/TP1.sql}
						Résultat : \texttt{Table created.}
				\item Relation \texttt{PATIENT}
					\lstinputlisting[firstline=45, lastline=52, style=OracleSQL]{../Scripts/TP1.sql}
						Résultat : \texttt{Table created.}
				\item Relation \texttt{HOSPITALISATION}
					\lstinputlisting[firstline=54, lastline=59, style=OracleSQL]{../Scripts/TP1.sql}
						Résultat : \texttt{Table created.}
				\item Relation \texttt{SOIGNE}
					\lstinputlisting[firstline=61, lastline=64, style=OracleSQL]{../Scripts/TP1.sql}
						Résultat : \texttt{Table created.}\\
				Ajout des contraintes
				\lstinputlisting[firstline=68, lastline=92, style=OracleSQL]{../Scripts/TP1.sql}
				Résultat : \texttt{Table altered.} (Après chaque requête)
			\end{enumerate}
		\item Ajouter l'attribut \texttt{DATE\_HOST} :
		\lstinputlisting[firstline=95, lastline=95, style=OracleSQL]{../Scripts/TP1.sql}
		Résultat : \texttt{Table altered.}
		\item Ajouter la contrainte \texttt{NOT NULL} aux 2 attributs :
		\lstinputlisting[firstline=98, lastline=99, style=OracleSQL]{../Scripts/TP1.sql}
		Résultat : \texttt{Table altered.}
		\item Modifier la longueur de \texttt{PRENOM\_PATIENT} :
		\lstinputlisting[firstline=102, lastline=103, style=OracleSQL]{../Scripts/TP1.sql}
		Résultat : \texttt{Table altered.}
		\item Supprimer et recréer la colonne \texttt{TEL\_EMP} :
		\lstinputlisting[firstline=106, lastline=108, style=OracleSQL]{../Scripts/TP1.sql}
		Résultats :\\
		\texttt{Table altered.}
		\begin{verbatim}
		Name                                      Null?    Type
		----------------------------------------- -------- ----------------------------
		NUM_EMP                                   NOT NULL NUMBER(4)
		NOM_EMP                                            NVARCHAR2(50)
		PRENOM_EMP                                         NVARCHAR2(50)
		ADRESSE_EMP                                        NVARCHAR2(150)
		\end{verbatim}
		\texttt{Table altered.}
		\item Renommer la colonne \texttt{ADRESSE\_PATIENT} :
		\lstinputlisting[firstline=111, lastline=112, style=OracleSQL]{../Scripts/TP1.sql}
		Résultats :\\
		\texttt{Table altered.}
		\begin{verbatim}
		Name                                      Null?    Type
		----------------------------------------- -------- ----------------------------
		NUM_PATIENT                               NOT NULL NUMBER(4)
		NOM_PATIENT                                        NVARCHAR2(50)
		PRENOM_PATIENT                                     NVARCHAR2(50)
		ADR_PAT                                            NVARCHAR2(150)
		MUTUELLE                                  NOT NULL NVARCHAR2(10)
		TEL_PATIENT                                        NUMBER(9)
		\end{verbatim}
		\item Ajouter la contrainte du salaire :
		\lstinputlisting[firstline=115, lastline=115, style=OracleSQL]{../Scripts/TP1.sql}
		Résultat : \texttt{Table altered.}
		\item Ajouter la contrainte qui impose que chaque médecin doit avoir une spécialité :
		\lstinputlisting[firstline=118, lastline=118, style=OracleSQL]{../Scripts/TP1.sql}
		Résultat : \texttt{Table altered.}
	\end{enumerate}
	\vspace{1em}
\section{Langage de manipulation de données}
	\begin{enumerate}
		\setcounter{enumi}{12}
		\item Remplir les instances :\\
		Nous devons commencer par insérer les valeurs du tableau \texttt{EMPLOYE} car ce tableau
		ne contient aucune clé étrangère, donc ses valeurs ne dépendent pas des valeurs d'autres
		tableaux. Ensuite, nous pouvons insérer les valeurs du tableau \texttt{MEDECIN} car la colonne \texttt{NUM\_MED}
		a une contrainte de clé étrangère vers la colonne \texttt{NUM\_EMP} du tableau \texttt{EMPLOYE}.
		Ensuite \texttt{SERVICE} car il dépend de \texttt{MEDECIN}. Ensuite \texttt{INFIRMIER} car il
		dépend de \texttt{SERVICE}. Ensuite \texttt{CHAMBRE} car il dépend de \texttt{INFIRMIER}. L'insertion
		des valeurs du tableau \texttt{HOSPITALISATION} doit être précédée par l'insertion des valeurs
		des tableaux \texttt{PATIENT}, \texttt{SERVICE}, et \texttt{CHAMBRE}. Les valeurs du tableau
		\texttt{SOIGNE} doivent être insérées après l'insertion des valeurs des tableaux \texttt{MEDECIN}
		et \texttt{PATIENT}.\\
		Les instructions \texttt{INSERT INTO} correspendantes sont dans le script \texttt{\bfseries TP1.sql}.
		\item Les étapes nécessaires pour que "Murray Andy" soit le nouveau directeur de Cardiologie :
		\begin{enumerate}
			\item Ajouter une ligne dans le tableau \texttt{EMPLOYE} contenant le numéro, le nom, le prénom,
			l'adresse et le téléphone de ce medecin.
			\item Ajouter une ligne dans le tableau \texttt{MEDECIN} contenant le numéro de ce medecin et
			 \texttt{'Cardiologue'}.
			\item Mettre à jour la colonne \texttt{DIRECTEUR} dans le tableau \texttt{SERVICE} au niveau de
			la ligne qui à la colonne \texttt{CODE\_SERVICE} égale à \texttt{'CAR'}.
		\end{enumerate}
		\lstinputlisting[firstline=561, lastline=563, style=OracleSQL, extendedchars=true]{../Scripts/TP1.sql}
		\item Diminuer 5000 du salaire :
		\lstinputlisting[firstline=567, lastline=567, style=OracleSQL, extendedchars=true]{../Scripts/TP1.sql}
		Resultat :
		\begin{verbatim}
		UPDATE INFIRMIER SET SALAIRE = SALAIRE - 5000 WHERE ROTATION = 'JOUR'
		*
		ERROR at line 1:
		ORA-02290: check constraint (ABBAD.CHK_SALAIRE) violated
		\end{verbatim}
		Désactiver la contrainte :
		\lstinputlisting[firstline=568, lastline=568, style=OracleSQL, extendedchars=true]{../Scripts/TP1.sql}
		Resultat : \texttt{Table altered.}\\
		Réessayer :
		\lstinputlisting[firstline=567, lastline=567, style=OracleSQL, extendedchars=true]{../Scripts/TP1.sql}
		Resultat : \texttt{14 rows updated.}\\
		Réactiver la contrainte
		\lstinputlisting[firstline=569, lastline=569, style=OracleSQL, extendedchars=true]{../Scripts/TP1.sql}
		Resultat :
		\begin{verbatim}
		ALTER TABLE INFIRMIER ENABLE CONSTRAINT CHK_SALAIRE
		                                        *
		ERROR at line 1:
		ORA-02293: cannot validate (ABBAD.CHK_SALAIRE) - check constraint violated
		\end{verbatim}
		\item Supprimer les cardiologues :
		\lstinputlisting[firstline=572, lastline=572, style=OracleSQL, extendedchars=true]{../Scripts/TP1.sql}
		Resultat :
		\begin{verbatim}
		DELETE FROM MEDECIN WHERE TRIM(SPECIALITE) = 'Cardiologue'
		*
		ERROR at line 1:
		ORA-01407: cannot update ("ABBAD"."SOIGNE"."NUM_MED") to NULL
		\end{verbatim}
		Ce n'est pas possible de supprimer ces lignes car certaines valeurs de la colonne \texttt{NUM\_MED} du
		tableau \texttt{SOIGNE} dépend des valeurs de la colonne \texttt{NUM\_MED} de \texttt{MEDECIN}.
	\end{enumerate}
\end{document}