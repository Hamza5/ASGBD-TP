\documentclass[12pt,a4paper]{article}
\usepackage[T1]{fontenc}
\usepackage[utf8]{inputenc}
\usepackage{lmodern}
\usepackage[frenchb]{babel}
\usepackage[left=1.5cm,right=1.5cm,top=2cm,bottom=2cm]{geometry}
\usepackage{listings}
\usepackage[usenames,dvipsnames]{xcolor}
\author{Hamza Abbad}
\title{Rapport de TP ASGBD}
\lstdefinestyle{OracleSQL}{language=SQL,breaklines=true, frame=single, keywordstyle=\bfseries\color{Blue},
keywordstyle=[2]\color{Blue},stringstyle=\color{OliveGreen},extendedchars=false,sensitive=false,
commentstyle=\slshape\color{Gray},basicstyle=\small\ttfamily,
morekeywords={TABLESPACE,AUTOEXTEND,TEMPORARY,PRIVILEGES,DATAFILE,IDENTIFIED,TEMPFILE,TO,ONLINE,USER,NVARCHAR2,REFERENCES,MODIFY}
}
\begin{document}
\section{Création des TableSpaces et des utilisateurs}
	\begin{enumerate}
		\item
		\begin{enumerate}
			\item Requête : \lstinputlisting[language=SQL, firstline=2, lastline=2, style=OracleSQL]{../Scripts/TP1.sql}
				Résultat : \texttt{SQL> Tablespace created.}
			\item Requête : \lstinputlisting[language=SQL, firstline=3, lastline=3, style=OracleSQL]{../Scripts/TP1.sql}
			Résultat : \texttt{Tablespace created.}
		\end{enumerate}
		\item Requête : \lstinputlisting[language=SQL, firstline=4, lastline=4, style=OracleSQL]{../Scripts/TP1.sql}
			Résultat : \texttt{User created.}
		\item Requête : \lstinputlisting[language=SQL, firstline=5, lastline=5, style=OracleSQL]{../Scripts/TP1.sql}
			Résultat : \texttt{Grant succeeded.}
	\end{enumerate}
\section{Langage de définition de données}
	\begin{enumerate}
		\setcounter{enumi}{3}
		\item Les clés étrangères :\\
			\renewcommand{\arraystretch}{1.2}
			\begin{tabular}{|*{4}{p{4cm}|}}
			\hline
			\multicolumn{2}{|c|}{Tableau source} & \multicolumn{2}{|c|}{Tableau destination} \\ \hline
			Dans le tableau & colonne & Vers le tableau & colonne \\ \hline
			SERVICE & DIRECTEUR & MEDECIN & NUM-MED \\ \hline
			CHAMBRE & CODE-SERVICE & SERVICE & CODE-SERVICE \\ \hline
			CHAMBRE & SURVEILLANT & INFERMIER & NUM-INF \\ \hline
			MEDECIN & NUM-MED & EMPLOYE & NUM-EMP \\ \hline
			INFERMIER & CODE-SERVICE & SERVICE & CODE-SERVICE \\ \hline
			INFERMIER & NUM-INF & EMPLOYE & NUM-EMP \\ \hline
			HOSPITALISATION & NUM-PATIENT & PATIENT & NUM-PATIENT \\ \hline
			HOSPITALISATION & CODE-SERVICE & SERVICE & CODE-SERVICE \\ \hline
			HOSPITALISATION & CODE-SERVICE, NUM-CHAMBRE & CHAMBRE & CODE-SERVICE \\ \hline
			SOIGNE & NUM-PATIENT & PATIENT & NUM-PATIENT \\ \hline
			\end{tabular}
		\item Création des relations
			\begin{enumerate}
				\item Relation \texttt{SERVICE}
					\lstinputlisting[firstline=11, lastline=16, style=OracleSQL]{../Scripts/TP1.sql}
					Résultat : \texttt{Table created.}
				\item Relation \texttt{CHAMBRE}
					\lstinputlisting[firstline=18, lastline=23, style=OracleSQL]{../Scripts/TP1.sql}
					Résultat : \texttt{Table created.}
				\item Relation \texttt{EMPLOYE}
					\lstinputlisting[firstline=25, lastline=31, style=OracleSQL]{../Scripts/TP1.sql}
						Résultat : \texttt{Table created.}
				\item Relation \texttt{MEDECIN}
					\lstinputlisting[firstline=33, lastline=36, style=OracleSQL]{../Scripts/TP1.sql}
						Résultat : \texttt{Table created.}
				\item Relation \texttt{INFERMIER}
					\lstinputlisting[firstline=38, lastline=43, style=OracleSQL]{../Scripts/TP1.sql}
						Résultat : \texttt{Table created.}
				\item Relation \texttt{PATIENT}
					\lstinputlisting[firstline=45, lastline=52, style=OracleSQL]{../Scripts/TP1.sql}
						Résultat : \texttt{Table created.}
				\item Relation \texttt{HOSPITALISATION}
					\lstinputlisting[firstline=54, lastline=59, style=OracleSQL]{../Scripts/TP1.sql}
						Résultat : \texttt{Table created.}
				\item Relation \texttt{SOIGNE}
					\lstinputlisting[firstline=61, lastline=64, style=OracleSQL]{../Scripts/TP1.sql}
						Résultat : \texttt{Table created.}\\
				Ajout des contraintes
				\lstinputlisting[firstline=68, lastline=90, style=OracleSQL]{../Scripts/TP1.sql}
				Résultat : \texttt{Table altered.} (Après chaque requête)
			\end{enumerate}
		\item Ajouter l'attribut \texttt{DATE\_HOST} :
		\lstinputlisting[firstline=93, lastline=93, style=OracleSQL]{../Scripts/TP1.sql}
		Résultat : \texttt{Table altered.}
		\item Ajouter la contrainte \texttt{NOT NULL} aux 2 attributs :
		\lstinputlisting[firstline=96, lastline=97, style=OracleSQL]{../Scripts/TP1.sql}
		Résultat : \texttt{Table altered.}
	\end{enumerate}
\end{document}