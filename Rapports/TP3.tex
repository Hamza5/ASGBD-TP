\documentclass[12pt,a4paper]{article}
\usepackage[T1]{fontenc}
\usepackage[utf8]{inputenc}
\usepackage{lmodern}
\usepackage[frenchb]{babel}
\usepackage[left=1.5cm,right=1.5cm,top=2cm,bottom=2cm]{geometry}
\usepackage{listings}
\usepackage[usenames,dvipsnames]{xcolor}
\usepackage{fancyhdr}
\author{Hamza ABBAD\\ Master Systèmes Informatiques Intelligents}
\title{\textbf{Rapport de TP ASGBD} \\ TP $n^{\circ}$3}
\lstdefinestyle{OracleSQL}{language=SQL,breaklines=true, frame=single, keywordstyle=\bfseries\color{Blue},
keywordstyle=[2]\color{Blue},stringstyle=\color{OliveGreen},extendedchars=false,sensitive=false,
commentstyle=\slshape\color{Gray},basicstyle=\small\ttfamily,
morekeywords={TABLESPACE,AUTOEXTEND,TEMPORARY,PRIVILEGES,DATAFILE,IDENTIFIED,TEMPFILE,TO,ONLINE,USER,NVARCHAR2,REFERENCES,
MODIFY,RENAME,DISABLE}
}
\pagestyle{fancyplain}
\lhead{Hamza ABBAD}
\chead{201200006342}
\rhead{SII M1 - Groupe 2}
\begin{document}
\maketitle
\begin{enumerate}
	\item Pour lister le catalogue \texttt{DICT} il suffit d'exécuter la requête suivante :
	\begin{verbatim}
		SELECT * FROM DICT;
	\end{verbatim}
	Elle retourne un grand nombre de lignes, dans mon cas, il a 1821 lignes. Il contient 2 colonnes :
	\begin{enumerate}
		\item \texttt{TABLE\_NAME} qui contient le nom de chaque tableau dans la méta-base.
		\item \texttt{COMMENTS} qui contient la description du contenu de ce tableau.
	\end{enumerate}
\end{enumerate}
\end{document}
